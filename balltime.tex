\documentclass{article}
\usepackage[style=ieee]{biblatex}
\usepackage{fontspec}
\usepackage{graphicx}
\usepackage{amsmath}
\usepackage{setspace}
\usepackage[document]{ragged2e}
\usepackage{listings}
\usepackage{textcomp}
\usepackage{siunitx}
\usepackage{fancyhdr}
\usepackage{unicode-math}

\DeclareSIUnit{\foot}{\text{ft}}
\DeclareSIUnit{\sper}{\text{/}}
\newcommand{\mcite}[1]{\ensuremath{\text{\hspace{0.2cm}\cite{#1}}}}

\addbibresource{balltime.bib}

\setmainfont{Roboto}
\setmathfont{Fira Math}
\setmonofont{AdwaitaMono Nerd Font}[Scale=0.75]

\title{Ball Drop Time Calculation}
\author{William Boyes}
\date{\today}

\pagestyle{fancy}
\fancyfoot[CO,CE]{\footnotesize \thepage \\ \copyright\ 2026, William Boyes, CC BY-NC-ND 4.0}

\begin{document}

\maketitle

\begin{center}
This work is licensed under CC BY-NC-ND 4.0. \\
To view a copy of this license, visit \\
https://creativecommons.org/licenses/by-nc-nd/4.0/
\end{center}

\newpage

\section{Initial calculations}

I am trying to calculate the exact time that it would take for \\
a smooth sphere (the ball) to reach the ground if free falling from \SI{10}{metre} \\
\vspace{1cm}
Our variables are \\

\begin{itemize}
    \item $a$ = acceleration
    \item $h$ = height
    \item $t$ = time, which our target is 0
\end{itemize}

For our case, acceleration will be equal to the acceleration of gravity \SI{9.8}{\metre\sper\second\squared}
And our initial height will be \SI{10}{\metre}

\vspace{1cm}

\begin{spacing}{2}
So we know the kinematic equation of $h = vt+\frac{1}{2}at^2$ \cite{wiki:kinequ} \\
from which we can figure out the formula $t = \sqrt{\frac{2h}{a}}$ \\
So we can fill in stuff + solve the $2h$ getting $\sqrt{\frac{20}{9.8}}$ \\
And then finally do $\sqrt{2.04} = \SI{1.428}{\second}$
\end{spacing}

So we know that the time for our item to fall to the ground starting from \SI{10}{\metre}, \\
WITHOUT factoring in air resistance is approximately \SI{1.428}{\second}. \\
But that's not exact, so now we're going to factor in air resistance. \\
Which is important because it can effect our times, \\
more info on drag force (air resistance) at \cite{nasa:whatisdrag}

\newpage
\section{Factoring in air resistance}

\begin{gather*}
\text{First we need to define our sizes} \\[1ex]
m = \SI{56}{\gram} = \SI{0.056}{\kilo\gram} \\
D = \SI{6.55}{\centi\metre} = \SI{0.0655}{\metre} \\
A = \pi(\frac{D}{2})^2 = 0.0033 \mcite{wiki:areacircle} \\[3ex]
\text{Now we'll define the environment we're in} \\[1ex]
T \approx \SI{10}{\celsius} = \SI{283.15}{\kelvin} \\
E \approx \SI{30}{\foot} \\
\rho \approx 1.2466 \mcite{wiki:airdens} \\[3ex]
\text{Almost there, just need to define our forces} \\[1ex]
C_d \approx 0.47 \SI{0.2}{\centi\metre} \mcite{wiki:dragco} \\
a_g = 9.8 \\[2ex]
F_g = ma_g = (0.056)(9.8) \approx 0.549 \\
F_d = \frac{1}{2}\rho v^2C_dA = 0.6233(v^2)0.47\times0.0033 = 0.00096v^2 \mcite{wiki:dragforce} \\
F_n = F_g - F_d = 0.549 - 0.00096v^2 = 0.549 \\[3ex]
\text{Finally we define our formulas which we'll use for calculations} \\[1ex]
a = a_g - a_d = a_g - \frac{F_d}{F_g} = 9.8 - \frac{0.00096v^2}{0.549} = 9.8 - 0.00175v^2 \\
v = u + at \\
h = h - vt
\end{gather*}

\newpage
\section{Final calculations}
We will do our first set of calculations in python, we will use $\Delta t = \SI{0.01}{\second}$

\begin{lstlisting}
h = 10.0
v = 0.0
t = 0.0
dt = 0.01
while h > 0:
    a = 9.8 - 0.00175 * v**2
    v += a * dt
    h -= v * dt
    t += dt
    print(f"|{t:.2f}|{a:.2f}|{v:.2f}|{h:.2f}|")
\end{lstlisting}

\begin{table}[h]
\centering
\begin{tabular}{cccc}
\hline
$t$ & $a$ & $v$ & $h$ \\
\hline
0.00 & 9.8 & 0 & 10 \\
0.01 & 9.80 & 0.10 & 10.00 \\
0.02 & 9.80 & 0.20 & 10.00 \\
0.03 & 9.80 & 0.29 & 9.99 \\
0.04 & 9.80 & 0.39 & 9.99 \\
... & ... & ... & ... \\
1.39 & 9.49 & 13.48 & 0.52 \\
1.40 & 9.48 & 13.57 & 0.38 \\
1.41 & 9.48 & 13.66 & 0.24 \\
1.42 & 9.47 & 13.76 & 0.11 \\
1.43 & 9.47 & 13.85 & -0.03 \\
\hline
\end{tabular}
\end{table}

WAIT we hit negative height, that's bad so we need to go back a tiny bit. We'll use python again with a smaller time step ($\Delta t = 0.001$) \\
and an accuracy of 3 decimals. That changes our table to this:

\begin{table}[h]
\centering
\begin{tabular}{cccc}
\hline
$t$ & $a$ & $v$ & $h$ \\
\hline
0.00 & 9.8 & 0 & 10 \\
0.001 & 9.800 & 0.010 & 10.000 \\
0.002 & 9.800 & 0.020 & 10.000 \\
0.003 & 9.800 & 0.029 & 10.000 \\
0.004 & 9.800 & 0.039 & 10.000 \\
... & ... & ... & ... \\
1.429 & 9.465 & 13.843 & 0.045 \\
1.430 & 9.465 & 13.853 & 0.031 \\
1.431 & 9.464 & 13.862 & 0.017 \\
1.432 & 9.464 & 13.872 & 0.003 \\
1.433 & 9.463 & 13.881 & -0.011 \\
\hline
\end{tabular}
\end{table}

\newpage

Just for the sake of time we'll go with $1.432$, a height of $0.003$ is good enough.

In conclusion, the time for it to hit the ground is around \SI{1.43}{\second}, \\
but if you want the real answer, I did more calculations to get

\begin{table}[h]
\centering
\begin{tabular}{cccc}
\hline
$t$ & $a$ & $v$ & $h$ \\
\hline
1.43274 & 9.46293 & 13.87838 & 0.00000 \\
\hline
\end{tabular}
\end{table}

Another thing I did was turn the data I got into a graph, so here it is:

\begin{figure}[h]
\centering
\includegraphics[width=0.8\textwidth]{results.png}
\caption{Graph of results}
\label{fig:results}
\end{figure}

\newpage
\printbibliography

\end{document}
